%=========================================
% 	   Einleitung     		 =
%=========================================
\chapter{Einleitung}

Seit Jahen ver"andern sich W"orter, manche W"orter bekommen eine andere Bedeutung, eine bestimmte Bedeutung bekommt ein neues Wort, wie z.B Ipod, ein altes Wort verschwindet komplett oder ein Wort bleibt "uber die Jahre gleich. Aber woher will man wissen welche W"orter sich ver"andert haben, dies hat niemand genau vermerkt, aber alle W"orter sind schon mal vorgekommen, n"amlich in den Zeitungen. Dann ist es ja einfach,  man muss nur alle Zeitungen zusammen suchen und die W"orter miteinander vergleichen, das Problem ist nur, dass es sehr viele Zeitungen gibt. Und um alle Zeitungen zu vergleichen braucht man mehrere Jahre, deswegen gibt es eine einfache L"osung, man l"asst den Computer alles vergleichen. Die Frage ist nur, wie kann der Computer die W"orter vergleichen? Da der Computer nicht wie ein Mensch denken kann, wird es mit dem Vergleichen schwer. Also m"ussen wir erst die Frage kl"aren, wie kann der Computer wie ein Mensch denken?
Daf"ur m"usste man neuronale Netze kennen und es dann in ein k"unstliches neuronales Netz umwandlen k"onnen.  Darauf aufbauend kann man mit Hilfe des Word Embeddings Wortrepr"asentationen erkennen und die passenden Vektoren bilden.\\
Diese beiden Theorien werden im Kapitel Technische Grundlagen genauer erkl"art.\\
F"ur diese Arbeit benutze ich beide Methoden, um die Fragen zu l"osen, wann hat sich ein Wort ver"andert, welches Wort hat sich wann ver"andert und welche W"orter sind zueinander gleich bzw. "ahnlich.\\\\

Die Arbeit ist in 6 weitere Kapitel gegliedert.
Im n"achsten Kapitel stehen die Technischen Grundlagen. Im 3. Kapitel kommen die verwandten Arbeiten. Im 4. Kapitel kommt die Vorgehensweise der Arbeit. Das 5. Kapitel ist die Implementierung. Das 6. Kapitel ist die Evaluation der Ergebnisse. Und das letzte Kapitel ist das Fazit.


%Hier wird erklärt worum es in dieser Bachlor-Arbeit geht.



